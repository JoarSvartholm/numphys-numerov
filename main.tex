\documentclass[a4paper,11pt]{article}
\usepackage[utf8]{inputenc}            % Tekenkodning
\usepackage[T1]{fontenc}               % Fixa kopiering av texten
\usepackage[english]{babel}            % Språk (t.ex. Innehåll)
\usepackage{geometry}                  % Sidlayout m.m.
\usepackage{graphicx,epstopdf,float}   % Bilder
\usepackage{amsmath,amssymb,amsfonts}  % Matematik
\usepackage{enumerate}                 % Fler typer av listor
\usepackage{fancyhdr}                  % Sidhuvud/sidfot
\usepackage{hyperref}                  % Hyperlänkar
\usepackage{parskip}                   % noindent!!
\usepackage{float}                     % \begin{figure}[H] preciserar bildposition
\usepackage{subcaption}             % För att lägga figurer bredvidvarandra från: http://tex.stackexchange.com/questions/91224/placing-two-figures-side-by-side
\usepackage{textcomp}
\usepackage{gensymb}
\usepackage{wrapfig}
\usepackage[]{algorithm2e}
% packages for matlab codes
\usepackage{listings}
\usepackage{color}
\usepackage{pdflscape}
\usepackage{multicol}
\setlength{\columnsep}{0.6cm}
% Add new commands used-defined:
\newcommand{\m}[1]{\mathbf{#1}}
\newcommand{\tn}[1]{\textnormal{#1}}
\newcommand{\ve}[1]{\textnormal{vec}(#1)}
\newcommand{\dn}{\textnormal{d}}


% instälningar för figurtexter
%\usepackage[margin=3ex,font=small,labelfont=bf,labelsep=endash]{caption}
\usepackage[font={small,it}]{caption}
\usepackage[labelfont={normal,bf}]{caption}
\usepackage[margin=3ex]{caption}

% mailadresser som hyperlänkar
\newcommand{\mail}[1]{\href{mailto:#1}{\nolinkurl{#1}}}
% Spara författare och titel
\let\oldAuthor\author
\renewcommand{\author}[1]{\newcommand{\myAuthor}{#1}\oldAuthor{#1}}
\let\oldTitle\title
\renewcommand{\title}[1]{\newcommand{\myTitle}{#1}\oldTitle{#1}}

% Hyperlänkar
\hypersetup{
  colorlinks   = true, %Colours links instead of ugly boxes
  urlcolor     = black, %Colour for external hyperlinks
  linkcolor    = black, %Colour of internal links
  citecolor   = black  %Colour of citations
}





\graphicspath{{./matlab/},{./figs/},{./matlab/figs/}} % Söker också bilder i en undermapp figs.


%% DOCUMENT
%------------------------------------------------------------------%
\begin{document}
  \title{Numerical solution of the time-independent Schrödinger equation using Numerov's method}


  \author{
    Joar Svartholm - josv0150(\mail{josv0150@student.umu.se})\\
  }
  \date{\today}


\begin{titlepage}
  \maketitle
  \thispagestyle{fancy}
  \headheight 35pt
  \rhead{\small\today}
  \lhead{\small Department of Physics\\
    Umeå Universitet}



% State the aim of the experiment, what was measured, which techniques and methods were used, and the main result(s) and conclusion(s). Remember that the abstract should be understandable on its own, and you can thereby not refer to equations/figures/tables in the report. You should also not use references, since the information in the abstract should be available in the actual report.

  % Ändra till rätt namn m.m.
  \cfoot{Numerical Methods in Physics \\
  Supervisor: Claude Dion}

\end{titlepage}


\newpage
\pagestyle{fancy}
\headheight 30pt
\rhead{\small Numerical solution of the time-\\independent Schrödinger equation using Numerov's method}
\lhead{\small \myAuthor \today}
\cfoot{\thepage}

% Innehåll
\tableofcontents
\newpage


\section{Time independent Schrödinger equation}

The time independent Shrödinger equation(TISE) is defined as

\begin{equation}
  \label{eq:TISE}
  \left[ -\frac{\hbar^2}{2m} \frac{\dn^2}{\dn x^2} + \mathcal V(x) \right]\Psi(x) = E\Psi(x) \Longleftrightarrow \frac{\dn^2}{\dn x^2}\Psi(x) + \frac{2m}{\hbar^2}\left[E - \mathcal V(x) \right] \Psi(x) = 0.
\end{equation}

Thus, by defining

\begin{subequations}
  \begin{align}
    k_{j}^2 &= \frac{2m}{\hbar^2}(E-\mathcal V(x_j)) \\
    Y_j &= \left(1 + \frac{m h^2}{6 \hbar^2}(E-\mathcal V(x_j))\right) \Psi(x_j)
  \end{align}
\end{subequations}

in Eq. \eqref{eq:TISE}, Numerov's iteration scheme will be

\begin{equation}
  \label{eq:numerovTISE}
  Y_{j+1} = Y_j \left( 2 - \frac{2m h^2}{\hbar^2} \frac{(E-\mathcal V(x_j))}{1+ \frac{mh^2}{6\hbar^2}(E-\mathcal V(x_j))} \right) - Y_{j-1}
\end{equation}


\section{The harmonic oscillator}

Consider now the potential of an harmonic oscillator

\begin{equation}
  \label{eq:harm}
  \mathcal V(x) = \frac{m \omega^2}{2}x^2.
\end{equation}

We will now find the wave function for the four first eigenstates using simulation units $\hbar = m = \omega = 1$. That is, for $E= \{0.5,1.5,2.5,3.5\}$. 


\section{The Morse potential}

Let us now consider the Morse potential

\begin{equation}
  \label{eq:morse}
  \mathcal V(R) = E_B \left[ 1-e^{-a(R-R_e)} \right]^2
\end{equation}

where is the distance between two atoms, $R_e$ is the equillibrium distance, $E_B$ is the binding energy and $a$ is a scaling constant for the bond stiffness. In the center of mass frame, it is better to use spherical coordinates for the wave function, $\Psi(\m R) = \Psi(R)Y(\theta,\phi)$. Taking the first orbital $Y = Y_0^0 = 1/\sqrt{4\pi}$, the problem reduces to a one dimensional problem since the orbital derivatives in the spherical laplacian derives to zero. Using the substitution $\psi (R) = R\Psi(R)$, TISE will reduce to

\begin{equation}
  \label{eq:spherTISE}
  \left[ -\frac{\hbar^2}{2\mu} \frac{\dn^2}{\dn R^2} + \mathcal V(R) \right]\psi(R) = E\psi(R),
\end{equation}

where $\mu$ is the reduced mass and $\mathcal V(R)$ is the potential in Eq. \eqref{eq:morse}.

Since the energy levels are not known in opposite to the harmonic potential a good guess for the energy levels are needed in order to find the correct eigen energies and its corresponding wavefunction. In order to do that we make a Taylor expansion of Eq. \eqref{eq:morse} around the equillibrium distance $R_e$

\begin{equation}
  \label{eq:taylor}
  \mathcal V(R) = \mathcal V(R_e) + (R-R_e)\left(\frac{\dn \mathcal V}{\dn R}\right)_{R_e} + \frac{(R-R_e)^2}{2}\left(\frac{\dn^2 \mathcal V}{\dn R^2}\right)_{R_e} + \mathcal O((R-R_e)^3).
\end{equation}

At equilibrium, both the function and its derivative is zero, so only the third term is nonzero and thus, by neglecting higher order terms the potential can be approximated as

\begin{equation}
  \label{eq:morseApprox}
  \begin{split}
  \mathcal V(R) &\approx \frac{(R-R_e)^2}{2} E_B \frac{\dn^2}{\dn x^2} \left[ 1-e^{-ax} \right]^2|_{x=0} \\
   & = \frac{(R-R_e)^2}{2} E_B 2 \frac{\dn}{\dn x} \left[1-e^{-ax}\right] a e^{-ax} |_{x=0}\\
   & = a(R-R_e)^2 E_B \left[ -a e^{-ax} +2a e^{-ax} \right]_{x=0} \\
   & = E_B a^2 (R-R_e)^2
  \end{split}
\end{equation}

when comparing this to the harmonic potential Eq. \eqref{eq:harm} it is clear that

\begin{equation*}
  \frac{\mu \omega^2}{2} = E_B a^2
\end{equation*}

must hold and the harmonic frequency can thus be written as

\begin{equation}
  \label{eq:omega}
  \omega = a \sqrt{\frac{2 E_B}{\mu}}
\end{equation}

For the harmonic oscillator, we know that the eigen energies are

\begin{equation*}
  E_n = \hbar \omega \left(n + \frac{1}{2} \right).
\end{equation*}

For the anharmonic oscillator we are considering, we can approximate the eigen energies using Eq. \eqref{eq:omega} and replacing the mass with the reduced mass to obtain

\begin{equation}
  \label{eq:enegiesMorse}
  \tilde{E_n} = \hbar a \sqrt{\frac{2 E_B}{\mu}} \left( n + \frac{1}{2}\right).
\end{equation}

The ground state energy and the spacing is thus

\begin{equation*}
  \begin{split}
    \tilde{E_0} & = \hbar a \sqrt{\frac{ E_B}{2\mu}} \\
    \Delta \tilde{E} & = \hbar a \sqrt{\frac{2 E_B}{\mu}}
  \end{split}
\end{equation*}



\section{End words}

All C code is found in the code folder;\newline \verb|/home/josv0150/Documents/numPhys/numphys-numerov/code| including the code for plotting. If something needs to be recompiled simply type \verb| make harm2 | or \verb|make morse2| for the harmonic and morse potential case. In order to run a program an extra argument should be written corresponding to an integer of which state that should be computed. In the morse program, and additional arbritary argument can be written in order to get only the energy of the corresponding state. The plot functions are written in python3 and the line \verb|python3 plotNumerov.py| or equivalent should do the trick. I have not tested to run these plotfunctions on sesam since I uploaded the codes remotely and I have not tried it for other python versions either.



\end{document}
